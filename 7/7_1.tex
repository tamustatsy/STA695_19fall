\documentclass{beamer}

\mode<presentation> {

%\usetheme{default}
%\usetheme{AnnArbor}
%\usetheme{Antibes}
%\usetheme{Bergen}
%\usetheme{Berkeley}
%\usetheme{Berlin}
%\usetheme{Boadilla}
%\usetheme{CambridgeUS}
%\usetheme{Copenhagen}
%\usetheme{Darmstadt}
%\usetheme{Dresden}
%\usetheme{Frankfurt}
%\usetheme{Goettingen}
%\usetheme{Hannover}
%\usetheme{Ilmenau}
%\usetheme{JuanLesPins}
%\usetheme{Luebeck}
\usetheme{Madrid}
%\usetheme{Malmoe}
%\usetheme{Marburg}
%\usetheme{Montpellier}
%\usetheme{PaloAlto}
%\usetheme{Pittsburgh}
%\usetheme{Rochester}
%\usetheme{Singapore}
%\usetheme{Szeged}
%\usetheme{Warsaw}


%\usecolortheme{albatross}
%\usecolortheme{beaver}
%\usecolortheme{beetle}
%\usecolortheme{crane}
%\usecolortheme{dolphin}
%\usecolortheme{dove}
%\usecolortheme{fly}
%\usecolortheme{lily}
%\usecolortheme{orchid}
%\usecolortheme{rose}
%\usecolortheme{seagull}
%\usecolortheme{seahorse}
%\usecolortheme{whale}
%\usecolortheme{wolverine}

%\setbeamertemplate{footline} % To remove the footer line in all slides uncomment this line
%\setbeamertemplate{footline}[page number] % To replace the footer line in all slides with a simple slide count uncomment this line

%\setbeamertemplate{navigation symbols}{} % To remove the navigation symbols from the bottom of all slides uncomment this line
}

\usepackage{graphicx} % Allows including images
\usepackage{booktabs} % Allows the use of \toprule, \midrule and \bottomrule in tables
\usepackage{amsfonts}
\usepackage{mathrsfs, bbold}
\usepackage{amsmath,amssymb,graphicx}
\usepackage{mathtools} % gather
\usepackage[export]{adjustbox} % right-aligned graphics

\makeatletter
\newcommand*\rel@kern[1]{\kern#1\dimexpr\macc@kerna}
\newcommand*\widebar[1]{%
  \begingroup
  \def\mathaccent##1##2{%
    \rel@kern{0.8}%
    \overline{\rel@kern{-0.8}\macc@nucleus\rel@kern{0.2}}%
    \rel@kern{-0.2}%
  }%
  \macc@depth\@ne
  \let\math@bgroup\@empty \let\math@egroup\macc@set@skewchar
  \mathsurround\z@ \frozen@everymath{\mathgroup\macc@group\relax}%
  \macc@set@skewchar\relax
  \let\mathaccentV\macc@nested@a
  \macc@nested@a\relax111{#1}%
  \endgroup
}
\makeatother

%----------------------------------------------------------------------------------------
%	TITLE PAGE
%----------------------------------------------------------------------------------------

\title["7"]{7: Evaluating, comparing and expanding models}

% \author{Taylor} 
% \institute[UVA] 
% {
% University of Virginia \\
% \medskip
% \textit{} 
% }
\date{10/14/19} 

\begin{document}
%----------------------------------------------------------------------------------------

\begin{frame}
\titlepage 
\end{frame}

%----------------------------------------------------------------------------------------
\begin{frame}
\frametitle{Introduction}

This chapter focuses mostly on quantifying a model's predictive capabilities for the purposes of model selection and expansion. 

\end{frame}

%----------------------------------------------------------------------------------------
\begin{frame}
\frametitle{New Notation!}

\begin{enumerate}
\item $f$ is the true model 
\item $y$ is the data we use to estimate our model
\item $\tilde{y}$ is the future (time series) or alternative (not time series) data that we test our predictions on
\item $p_{\text{post}}(\tilde{y}) = p(\tilde{y} \mid y )$
\item $p_{\text{post}}(\theta) = p(\theta \mid y)$
\item $E_{\text{post}}[ \cdot ] $ is taken with respect to $p(\theta \mid y)$
\end{enumerate}


\end{frame}


%----------------------------------------------------------------------------------------
\begin{frame}
\frametitle{Definitions}

A {\bf scoring rule/function} $S(p,\tilde{y})$ is a function that takes
\begin{enumerate}
\item the distribution you're using to forecast $p$ (ppd, or likelihood with estimated parameters), and 
\item a realized value $\tilde{y}$
\end{enumerate}
and then gives you a real-valued number/score/utility. Higher is better, although this convention isn't always followed in the literature.
\newline

Keep in mind that the realized value cannot be used to fit the data.
\end{frame}

%----------------------------------------------------------------------------------------
\begin{frame}
\frametitle{Examples}


Example: $S(p,\tilde{y}) = -(\tilde{y} - E_p[\tilde{y}])^2$
\newline


Example: $S(p,\tilde{y}) = \log p(\tilde{y})$
\newline


\end{frame}

%----------------------------------------------------------------------------------------
\begin{frame}
\frametitle{(Out-of-sample) predictive fit}

Future/unseen data is unknown, so we must take the expected score under the true distribution $f$:
$$
E_f[S(p,\tilde{y})].
$$

A scoring rule is {\bf proper} if the above expectation is maximized when $f = p$.
\newline

A scoring rule is {\bf local} if $S(p,\tilde{y})$ only depends on $p(\tilde{y})$ (don't care about events that didn't happen).
\newline

Note, when we are dealing with a logarithmic scoring rule, $E[-2\log p(\tilde{y})]$ is often called an {\bf information criterion.} The book switches back and forth between dealing with expected score, and information criteria. 

\end{frame}

%----------------------------------------------------------------------------------------
\begin{frame}
\frametitle{Examples}


Example: $S(p,\tilde{y}) = -(\tilde{y} - E_p[\tilde{y}])^2$ \\
Most common, perhaps not local or proper for non-Gaussian data.
\newline

Example: $S(p,\tilde{y}) = \log p(\tilde{y})$\\
Obviously local. Proper, too.


\end{frame}


%----------------------------------------------------------------------------------------
\begin{frame}
\frametitle{Empirical predictive fit}

% We may predict data with the ppd, or plug some point estimate $\hat{\theta}$ into the likelihood. 
% \newline

We are generally not able to evaluate the expectation because we don't know $f$. However, we may be able to wait for new out-of-sample data and use a Monte-Carlo approach:
\[
n^{-1}\sum_{i=1}^{n} S(p,\tilde{y}^i) \to E_f[S(p,\tilde{y})]
\]
as $n \to \infty$
\newline
\pause

% If we can afford to wait for an infinite amount of data, though, what is the point of trying to predict it?

\end{frame}



%----------------------------------------------------------------------------------------
\begin{frame}
\frametitle{Definitions}

The textbook focuses on $S(p,\tilde{y}) = \log p(\tilde{y})$, and the data are iid (after conditioning on the parameter). They call the following quantity the ``elppd:"

\begin{block}{expected log pointwise predictive density}
\begin{align*}
E_f [\log p(\tilde{y})] &= E_f\left[\log \prod_i p(\tilde{y}_i) \right] \\
&= \sum_{i=1}^nE_f\left[ \log p(\tilde{y}_i) \right]
\end{align*}
\end{block}

\begin{itemize}
\item In general, $E_f [\log p(\tilde{y})] \neq \sum_{i=1}^nE_f\left[
    \log p(\tilde{y}_i) \right]$. Sum of ``pointwise" ppd not equal to
    ``joint" ppd.
\end{itemize}

\end{frame}

%----------------------------------------------------------------------------------------
\begin{frame}
\frametitle{Problem}

For the moment let's use $p(\tilde{y}) = p_{\text{post}}(\tilde{y})$
\newline

The ``elppd" is not obtainable because
\begin{enumerate}
\item you don't know $f$ 
\item you don't have $p_{\text{post}}(\tilde{y})$ 
\end{enumerate}
\pause

Using $y$ for $\tilde{y}$, we can come up with a rough elppd estimate called the ``lppd"
\begin{block}{log pointwise predictive density}
\[
\text{lppd} = \log p_{\text{post}}(y) = \sum_{i=1}^n \log p_{\text{post}}(y_i) 
\]
\end{block}



\end{frame}


%----------------------------------------------------------------------------------------
\begin{frame}
\frametitle{Problem}

There'a also the problem that arises where we cannot evaluate 
\[
p_{\text{post}}(y) = \int p(y \mid \theta) p(\theta \mid y) \text{d}\theta = E_{\text{post}}[p(y \mid \theta)]
\]

The ``computed lppd" again uses $y$ for $\tilde{y}$, but it also uses Monte-Carlo to sample from the posterior 
\begin{block}{log pointwise predictive density}
\[
\text{computed lppd} = \log \hat{p}_{\text{post}}(y) = \sum_{i=1}^n \log \left( \frac{1}{S} \sum_{j=1}^S p(y_i \mid \theta^j) \right) 
\]
\end{block}

-Biased and probably high variance, though.

\end{frame}



%----------------------------------------------------------------------------------------
\begin{frame}
\frametitle{Three problems}

Don't know $f$, don't want to wait for $\tilde{y}$...
\newline

and unfortunately, plugging the same data that we used for estimation into the predictive distribution might lead us to overfit because this strategy overestimates the average predictive score. What do we do?
\newline
\pause

However, we can get around this in two ways generally:
\begin{enumerate}
\item plug in the already-used $y$ data, but then add an extra penalty term (e.g. AIC, DIC, WAIC, etc.)
\item Cross-Validation: split the data $y$, many different ways, into a train and test set; estimate and evaluate on each split.
\end{enumerate}

\end{frame}

%----------------------------------------------------------------------------------------
\begin{frame}
\frametitle{Information Criteria}

{\bf AIC} stands for ``Akaike's Information Criterion." Let $k$ be the number of parameters:
\newline

\[
\widehat{\text{elpd}}_{\text{AIC}} = \log p(y \mid \hat{\theta}_{\text{MLE}}) - \overbrace{k}^{\text{penalty}}
\]
or
\[
\text{AIC} = \underbrace{-2\log p(y \mid \hat{\theta}_{\text{MLE}})}_{\text{a deviance}} +2 k
\]

We estimate $\hat{\theta}_{\text{MLE}}$ using $y$, \*and\* we plug $y$ into the log likelihood. 

\end{frame}


%----------------------------------------------------------------------------------------
\begin{frame}
\frametitle{Information Criteria}

{\bf DIC} replaces the point estimate with $\hat{\theta}_{\text{Bayes}} = E[\theta \mid y]$, and replaces the penalty term with $p_{\text{DIC}}$
\newline

\[
\widehat{\text{elpd}}_{\text{DIC}} = \log p(y \mid \hat{\theta}_{\text{Bayes}}) - p_{\text{DIC}}
\]
or
\[
\text{DIC} = -2\log p(y \mid \hat{\theta}_{\text{Bayes}}) +2 p_{\text{DIC}}
\]

\end{frame}

%----------------------------------------------------------------------------------------
\begin{frame}
\frametitle{Information Criteria}

The book gives two ways to estimate $p_{\text{DIC}}$:

\begin{enumerate}
\item $p_{\text{DIC}} = 2\left(\log p(y \mid \hat{\theta}_{\text{Bayes}}) - E_{\text{post}}\left[ \log p(y \mid \theta) \right] \right)$
\item $p_{\text{DIC alt}} = 2 \operatorname{Var}_{\text{post}}\left[ \log p(y \mid \theta) \right]$
\end{enumerate}

Both of these can be approximated using samples from the posterior.

\begin{itemize}
\item Both $p_{\text{DIC}}$ and $p_{\text{DIC alt}}$ are estimated
  effective number of parameters
\item Both reduce to $k$ for linear models with uniform prior distributions
\end{itemize}
\end{frame}

%----------------------------------------------------------------------------------------
\begin{frame}
\frametitle{Information Criteria}

Motivation for $p_{\text{DIC}}$
\begin{align*}
&E_{\tilde{y}}\left[-2  \log p(\tilde{y} \mid \hat{\theta}_{\text{Bayes}})  \right] \\
&= - 2\log p(y \mid \hat{\theta}_{\text{Bayes}}) + E_{\tilde{y}}\left[  -2\log p(\tilde{y} \mid \hat{\theta}_{\text{Bayes}})  \right] + 2\log p(y \mid \hat{\theta}_{\text{Bayes}})\\
&\approx - 2\log p(y \mid \hat{\theta}_{\text{Bayes}}) +  E_{\theta \mid y}\left[ - 2 \log p(y \mid \theta) \right] + 2 \log p(y \mid \hat{\theta}_{\text{Bayes}} ) \\
&= - 2\log p(y \mid \hat{\theta}_{\text{Bayes}}) + p_{\text{DIC}}
\end{align*}


\end{frame}


%----------------------------------------------------------------------------------------
\begin{frame}
\frametitle{Information Criteria}

$p_{\text{WAIC}}$ either stands for ``Watanabe-Akaike information criterion." 
\newline

The book refers to it as the most ``fully Bayesian" of the three, probably because it doesn't plug in point estimates into the likelihood instead of integrating.

\[
\widehat{\text{elppd}}_{\text{WAIC}} = \text{lppd} - p_{\text{WAIC}}
\]
or
\[
\text{WAIC} = -2\text{lppd} + 2 p_{\text{WAIC}}
\]

where $ \text{lppd}$ is computed by Monte Carlo method as $\sum_{i=1}^n \log \left(\frac{1}{S}\sum_{s=1}^S p(y_i \mid \theta^s) \right)$

\end{frame}

%----------------------------------------------------------------------------------------
\begin{frame}
\frametitle{Information Criteria}

Two ways to estimate 

\begin{enumerate}
\item $p_{\text{WAIC 1}} = 2\sum_{i=1}^n \left( \log E_{\text{post}}\{p(y_i \mid y)\} - E_{post} \left\{  \log p(y_i \mid \theta) \right\} \right)$
\item $p_{\text{WAIC 2}} = \sum_{i=1}^n var_{\text{post}}(\log p(y_i \mid \theta))$
\end{enumerate}

Both of these can be approximated using samples from the posterior.

\end{frame}


%----------------------------------------------------------------------------------------
\begin{frame}
\frametitle{Information Criteria}

Motivation for $p_{\text{WAIC}}$:

\begin{align*}
% &E_{\tilde{y}}\left[ -2  \log   p(\tilde{y} \mid y)  \right] \\
&E_{\tilde{y}}\left[ -2  \log  E_{\theta \mid y} ( p(\tilde{y} \mid \theta) ) \right] \\
&= - 2 lppd + 2\left( lppd - E_{\tilde{y}}\left[  \log  E_{\theta \mid y} ( p(\tilde{y} \mid \theta) ) \right] \right) \\
&\approx - 2 lppd + 2\left( lppd - E_{\tilde{y}}\left[   E_{\theta \mid y} ( \log  p(\tilde{y} \mid \theta) ) \right] \right) \\
&= - 2 lppd + 2\left( lppd - E_{\theta \mid y}\left[   E_{\tilde{y}} (\log  p(\tilde{y} \mid \theta) ) \right] \right) \\
&\approx - 2 lppd + 2\left( lppd - E_{\theta \mid y}\left[  \log   p(y \mid \theta)  \right] \right) \\
&= - 2 lppd + 2\left( \sum_i \log p_{\text{post}}(y_i) - E_{\text{post}}\left[  \log   \prod_i p(y_i \mid \theta)  \right] \right) \\
&= - 2 lppd + 2 \sum_i \left(  \log p_{\text{post}}(y_i) - E_{\text{post}}\left[ \log p(y_i \mid \theta)  \right] \right) \\
&= - 2\log p(y \mid y) + p_{\text{WAIC} 1}
\end{align*}



\end{frame}

%----------------------------------------------------------------------------------------
\begin{frame}
\frametitle{Cross-Validation}

To assess prediction performance, one may also use {\bf cross-validation}. Here the data is repeatedly partitioned into different training-set-test-set pairs (aka {\bf folds}).
\pause

\begin{enumerate}
\item The partitions are nonrandom, test sets are disjoint
\item for each split/estimation/prediction, we never use a data point twice
\item for each split/estimation/prediction, we lose parameter estimation accuracy because each training set is smaller than the full set
\item however, we get to average over many prediction scores, which reduces variance
\item there is still a bias that we have to estimate (due to small
  sample size but it's usually smaller than AIC/DIC/WAIC/etc.)
\item it can be computationally brutal to calculate for some models
\end{enumerate}
% \pause

% \includegraphics[width=50mm,right]{cv_logo.png}

% The logo of this QA website illustrates the idea nicely!
% \newline

\end{frame}

%----------------------------------------------------------------------------------------
\begin{frame}
\frametitle{Cross-Validation}


{\bf leave-one-out cross-validation} (loo-cv) is a special case where each test set is of size $1$.
\newline

This necessarily implies that each training set is of size $n-1$, and there are $n$ possible splits.
\newline

If this ends up being too computationally expensive, it is also possible to do {\bf $k$-fold cross-validation}, which selects $k$ splits/folds. This means the size of each test set is $n/k$, and the size of each training set is $n -n/k$

\end{frame}

%----------------------------------------------------------------------------------------
\begin{frame}
\frametitle{Cross-Validation Notation}

We only discuss loo-cv...
\newline

$p_{\text{post}(-i)}(y_i)$ is the prediction for the $i$th point, using the ppd, which uses the posterior distribution conditioning on all values of the data {\bf except the} $i$th
\newline
\pause

If this ppd isn't tractable, we can use draws from the posterior as follows:
\[
p_{\text{post}(-i)}(y_i) = \frac{1}{S}\sum_{s=1}^S p(y_i \mid \theta^s)
\]
where $\theta^{is}$ are draws from $p_{\text{post}(-i)}(\theta)$
\newline

\end{frame}

%----------------------------------------------------------------------------------------
\begin{frame}
\frametitle{Cross-Validation}


The Bayesian loo-cv estimate for out-of-sample predictive fit is 

\[
\text{lppd}_{\text{loo-cv}} = \sum_{i=1}^n \log p_{\text{post}(-i)}(y_i)
\]

There are also bias-corrected versions as well, that is,
$\text{lppd}_{\text{loo-cv}} + b$, where

\[ 
b = lppd - \widebar{lppd}_{-i}
\]
\[
\widebar{lppd}_{-i} = \frac{1}{n} \sum_{i=1}^n \sum_{j=1}^n
\log p_{\text{post}(-i)}(y_j)
\]
\end{frame}

\begin{frame}
  \frametitle{Comparisons of all comparison criteria}
Given score function $S(y, p) = \log p(y)$
  \begin{itemize}
  \item AIC and DIC are based on the {\bf joint likelihood} score function
    conditioning on a point estimate of $\theta$; WAIC and LOO-CV are
    based on {\bf individual posterior} score function averaging over the posterior distribution $p(\theta \mid
    y)$
\pause
\item WAIC and LOO-CV require an explicit assumption/requirement that data are
  independent conditioning on the parameter; not easy to do in some
  structured-data settings such as time series, spatial and network
  data
\pause
\item WAIC and LOO-CV are equal asymptotically 
\pause
\item AIC and LOO-CV are restrictive when applied to hierarchical
  models (eight school example)
\pause
  \end{itemize}

Overall, WAIC is more appealing than the rest of them.
\end{frame}

\end{document} 



%%% Local Variables:
%%% mode: latex
%%% TeX-master: t
%%% End:
