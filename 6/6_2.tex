\documentclass{beamer}

\mode<presentation> {

%\usetheme{default}
%\usetheme{AnnArbor}
%\usetheme{Antibes}
%\usetheme{Bergen}
%\usetheme{Berkeley}
%\usetheme{Berlin}
%\usetheme{Boadilla}
%\usetheme{CambridgeUS}
%\usetheme{Copenhagen}
%\usetheme{Darmstadt}
%\usetheme{Dresden}
%\usetheme{Frankfurt}
%\usetheme{Goettingen}
%\usetheme{Hannover}
%\usetheme{Ilmenau}
%\usetheme{JuanLesPins}
%\usetheme{Luebeck}
\usetheme{Madrid}
%\usetheme{Malmoe}
%\usetheme{Marburg}
%\usetheme{Montpellier}
%\usetheme{PaloAlto}
%\usetheme{Pittsburgh}
%\usetheme{Rochester}
%\usetheme{Singapore}
%\usetheme{Szeged}
%\usetheme{Warsaw}


%\usecolortheme{albatross}
%\usecolortheme{beaver}
%\usecolortheme{beetle}
%\usecolortheme{crane}
%\usecolortheme{dolphin}
%\usecolortheme{dove}
%\usecolortheme{fly}
%\usecolortheme{lily}
%\usecolortheme{orchid}
%\usecolortheme{rose}
%\usecolortheme{seagull}
%\usecolortheme{seahorse}
%\usecolortheme{whale}
%\usecolortheme{wolverine}

%\setbeamertemplate{footline} % To remove the footer line in all slides uncomment this line
%\setbeamertemplate{footline}[page number] % To replace the footer line in all slides with a simple slide count uncomment this line

%\setbeamertemplate{navigation symbols}{} % To remove the navigation symbols from the bottom of all slides uncomment this line
}

\usepackage{graphicx} % Allows including images
\usepackage{booktabs} % Allows the use of \toprule, \midrule and \bottomrule in tables
\usepackage{amsfonts}
\usepackage{mathrsfs, bbold}
\usepackage{amsmath,amssymb,graphicx}
\usepackage{mathtools} % gather

%----------------------------------------------------------------------------------------
%	TITLE PAGE
%----------------------------------------------------------------------------------------

\title["6"]{6: Model Checking}

\date{10/09/19} 

\begin{document}
%----------------------------------------------------------------------------------------

\begin{frame}
\titlepage 
\end{frame}


%----------------------------------------------------------------------------------------
\begin{frame}
\frametitle{Model checking and likelihood principle}  
Does the average score of $55$ students on a test exceed $80$? Assume
that the test scores are normally distributed $\mbox{N}(\mu,\sigma^2)$.
\newline

Imagine four ways of collecting the data:

\begin{itemize}
\item Random sample of $55$ students
\item Randomly sample students in sequence and after each student
  cease collecting data with probability $0.02$
\item Randomly sample students for a fixed amount of time
\item Continue to randomly sample until the sample mean is
  significantly different from $80$ using $t$-test
\end{itemize}

Among all four data sampling rules, the likelihood stays the same and thus $p(\theta \mid y)$. However,
$p(y^{rep} \mid y)$ is not. Model checking depends on
data-collection model.
\end{frame}

%----------------------------------------------------------------------------------------
\begin{frame}
\frametitle{Marginal posterior checks}
A (marginal) posterior predictive p-value based on the marginal
posterior predictive distribution $p(y_i \mid y)$.

\[
p_i = \mbox{Pr}(T(y_i^{rep}) \leq T(y_i) \mid y)
\]  

A natural test quantity is $T(y_i) = y_i$.
\newline 

For the ETS example, $y_i^{rep}$ is generated from $P(\tilde{y}_i\mid
y)$, $i = 1,\ldots,8$ or $P(\tilde{y}_i\mid
y)$, $i = 9, \ldots, 16$ depending on predictions are made for
existing schools or new schools. 
\newline 

In the first case, all separate $p_i$'s tend to concentrate around
$0.5$ while look more uniform in the second case. 
\end{frame}


%----------------------------------------------------------------------------------------
\begin{frame}
  \frametitle{Marginal posterior checks}
A cross validation predictive p-value:
\[
p_i = \mbox{Pr}(T(y_i^{rep}) \leq T(y_i) \mid y_{-i})
\]   

Cross validated p-values require more computation but are supposed to
be more uniform.
\end{frame}

%----------------------------------------------------------------------------------------
\begin{frame}
  \frametitle{What patterns can reveal }
  When we examine the collection of marginal p-values, $p_i$'s,
  \begin{itemize}
  \item Marginal posterior p-values concentrate near 0 or 1, data is
    over-dispersed compared to the model
  \item Marginal posterior p-values concentrate near 0.5, data is
    under-dispersed compared to the model 
  \end{itemize}
\end{frame}


%----------------------------------------------------------------------------------------
\begin{frame}
  \frametitle{Graphical posterior predictive checks}
  \begin{itemize}
  \item Direct display all the data (e.g. the speed of light
    data)
\item Display of data summaries (e.g. the speed of light data,
  validating the assumption in a Bernoulli experiment) or parameter
  inferences
\item Graphs of residuals or other measures of discrepancy between
  model and data
  \end{itemize}
\end{frame}


% %----------------------------------------------------------------------------------------
% \begin{frame}
% \frametitle{A Nice Graphic (the old book cover)}

% The next slide's picture plots predictions and real data from a psychological study.
% \newline

% $y = \{y_{i,j,k} \}$ is an array of binary variables, with 
% \begin{enumerate}
% \item $i=1,\ldots,6$ denoting the person
% \item $j = 1, \ldots, 15$ denoting which reaction (which row)
% \item $k = 1, \ldots, 23$ situation  (which column)
% \end{enumerate}

% \end{frame}

% %----------------------------------------------------------------------------------------
% \begin{frame}
% \frametitle{A Nice Graphic (the old book cover)}

% \begin{center}
% \includegraphics[width=100mm]{data_display.png}
% \end{center}


% \end{frame}
% %----------------------------------------------------------------------------------------
% \begin{frame}
% \frametitle{Without Sorting}

% \begin{center}
% \includegraphics[width=100mm]{unsorted.png}
% \end{center}


% \end{frame}

\end{document} 



%%% Local Variables:
%%% mode: latex
%%% TeX-master: t
%%% End:
