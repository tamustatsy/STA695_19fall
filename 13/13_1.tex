\documentclass{beamer}

\mode<presentation> {

%\usetheme{default}
%\usetheme{AnnArbor}
%\usetheme{Antibes}
%\usetheme{Bergen}
%\usetheme{Berkeley}
%\usetheme{Berlin}
%\usetheme{Boadilla}
%\usetheme{CambridgeUS}
%\usetheme{Copenhagen}
%\usetheme{Darmstadt}
%\usetheme{Dresden}
%\usetheme{Frankfurt}
%\usetheme{Goettingen}
%\usetheme{Hannover}
%\usetheme{Ilmenau}
%\usetheme{JuanLesPins}
%\usetheme{Luebeck}
\usetheme{Madrid}
%\usetheme{Malmoe}
%\usetheme{Marburg}
%\usetheme{Montpellier}
%\usetheme{PaloAlto}
%\usetheme{Pittsburgh}
%\usetheme{Rochester}
%\usetheme{Singapore}
%\usetheme{Szeged}
%\usetheme{Warsaw}


%\usecolortheme{albatross}
%\usecolortheme{beaver}
%\usecolortheme{beetle}
%\usecolortheme{crane}
%\usecolortheme{dolphin}
%\usecolortheme{dove}
%\usecolortheme{fly}
%\usecolortheme{lily}
%\usecolortheme{orchid}
%\usecolortheme{rose}
%\usecolortheme{seagull}
%\usecolortheme{seahorse}
%\usecolortheme{whale}
%\usecolortheme{wolverine}

%\setbeamertemplate{footline} % To remove the footer line in all slides uncomment this line
%\setbeamertemplate{footline}[page number] % To replace the footer line in all slides with a simple slide count uncomment this line

%\setbeamertemplate{navigation symbols}{} % To remove the navigation symbols from the bottom of all slides uncomment this line
}

\usepackage{graphicx} % Allows including images
\usepackage{booktabs} % Allows the use of \toprule, \midrule and \bottomrule in tables
\usepackage{amsfonts}
\usepackage{mathrsfs, bbold}
\usepackage{amsmath,amssymb,graphicx}
\usepackage{mathtools} % gather
\usepackage[export]{adjustbox} % right-aligned graphics

% argmax
\DeclareMathOperator*{\argmax}{arg\,max}

%----------------------------------------------------------------------------------------
%	TITLE PAGE
%----------------------------------------------------------------------------------------

\title["13"]{13: Modal And Distributional Approximations}

\author{Taylor} 
% \institute[UVA] 
% {
% University of Virginia \\
% \medskip
% \textit{} 
% }
\date{11/20/19} 

\begin{document}
%----------------------------------------------------------------------------------------

\begin{frame}
\titlepage 
\end{frame}

% %----------------------------------------------------------------------------------------
% \begin{frame}
% \frametitle{Introduction}

% We mention:
% \begin{enumerate}
% \item a few ways to find the posterior mode
% \item how to approximate a posterior using a mode 
% \item slightly more involved ways to approximate your posterior
% \end{enumerate}

% % For various reasons, we also frequently split up our parameters into two groups: $\theta = (\gamma, \phi)$.


% \end{frame}


%----------------------------------------------------------------------------------------
\begin{frame}
\frametitle{Approximations using mixture distributions}

The book has some discussions about approximating a posterior
distribution in a direct way (including finding the mode/modes) that we don't address:

\begin{enumerate}
\item approximating multimodal distributions with normal mixtures
\item approximating multimodal distributions with t mixtures
\end{enumerate}

\end{frame}


%----------------------------------------------------------------------------------------
\begin{frame}
\frametitle{The EM Algorithm}

The {\bf expectation-maximization algorithm} finds the argument that maximizes the marginal posterior. It's useful in situations where there is missing data in a model (e.g. hierarchical models, factor models, hidden markov models, state space models, etc.). 
\newline

It folows the following steps
\begin{enumerate}
\item replace missing values by their expectations given the guessed parameters, 
\item estimate parameters assuming the missing data are equal to their estimated values, 
\item re-estimate the missing values assuming the new parameter estimates are correct,
\item re-estimate parameters, 
\end{enumerate}
and so forth, iterating until convergence.

\end{frame}


%----------------------------------------------------------------------------------------
\begin{frame}[fragile]
\frametitle{The EM Algorithm}

Call $\theta = (\gamma, \phi)$. You're interested in the mode of $p(\phi \mid y)$. Typically, $\gamma$ is ``hidden data."
\newline

$$
\log p(\phi \mid y) = \log \frac{p(\gamma, \phi \mid y)}{p(\gamma \mid \phi, y)} = \log \underbrace{p(\gamma, \phi \mid y)}_{ \text{joint posterior} } - \log \underbrace{p(\gamma \mid \phi, y)}_{\text{conditional posterior }}
$$
\pause

taking expectations on both sides with respect to $p(\gamma \mid \phi^{\text{old}}, y)$ yields:
$$
\log p(\phi \mid y) =  E\left[ \log p(\gamma, \phi \mid y) \mid \phi^{\text{old}}, y \right] - E\left[\log p(\gamma \mid \phi, y) \mid \phi^{\text{old}}, y \right]
$$

% exercise 13.6 !



\end{frame}

%----------------------------------------------------------------------------------------
\begin{frame}
\frametitle{The EM Algorithm}

We iteratively use the middle term in
$\log p(\phi \mid y) =  E\left[ \log p(\gamma, \phi \mid y) \mid \phi^{\text{old}}, y \right] - E\left[\log p(\gamma \mid \phi, y) \mid \phi^{\text{old}}, y \right]$.

\begin{block}{The Q quantity in the ``E" step}

$$
Q(\phi \mid \phi^{\text{old}}) = E\left[ \log p(\gamma, \phi \mid y) \mid \phi^{\text{old}}, y \right]
$$
\end{block}
\pause

\begin{block}{The EM algorithm}

Repeat the following until convergence:
\begin{enumerate}
\item E-step: calculate $Q(\phi \mid \phi^{\text{old}})$
\item M-step: replace $\phi^{\text{old}}$ with $\argmax Q(\phi \mid \phi^{\text{old}})$
\end{enumerate}
\end{block}
\end{frame}

%----------------------------------------------------------------------------------------
\begin{frame}
\frametitle{The EM Algorithm}

If we follow this strategy, $\log p(\phi \mid y)$ increases at every iteration:
\begin{align*}
\log p(\phi \mid y) &=  E\left[ \log p(\gamma, \phi \mid y) \mid \phi^{\text{old}}, y \right] - E\left[\log p(\gamma \mid \phi, y) \mid \phi^{\text{old}}, y \right]\\
&= Q(\phi \mid \phi^{\text{old}}) - E\left[\log p(\gamma \mid \phi, y) \mid \phi^{\text{old}}, y \right] \tag{defn. Q} \\
&\ge Q(\phi \mid \phi^{\text{old}}) - E\left[\log p(\gamma \mid \phi^{\text{old}}, y) \mid \phi^{\text{old}}, y \right] \tag{HW}
\end{align*}
\pause

So 
\begin{align*}
&\log p(\phi^{\text{new}} \mid y) - \log p(\phi^{\text{old}} \mid y) \\
&= \log p(\phi^{\text{new}} \mid y) - \left\{Q(\phi^{\text{old}} \mid \phi^{\text{old}}) - E\left[\log p(\gamma \mid \phi^{\text{old}}, y) \mid \phi^{\text{old}}, y \right] \right\} \\
&\ge Q(\phi^{\text{new}} \mid \phi^{\text{old}}) - E\left[\log p(\gamma \mid \phi^{\text{old}}, y) \mid \phi^{\text{old}}, y \right] \\
&\hspace{10mm} - \left\{Q(\phi^{\text{old}} \mid \phi^{\text{old}}) - E\left[\log p(\gamma \mid \phi^{\text{old}}, y) \mid \phi^{\text{old}}, y \right] \right\} \\
&= Q(\phi^{\text{new}} \mid \phi^{\text{old}}) - Q(\phi^{\text{old}} \mid \phi^{\text{old}})
\end{align*}

\end{frame}

%----------------------------------------------------------------------------------------
\begin{frame}
\frametitle{The EM Algorithm}

Notes:
\begin{enumerate}
\item The EM algo isn't inherently Bayesian. It can also be used to accomplish maximum likelihood estimation.
\item The expectation of $\log p(\gamma, \phi \mid y)$ is usually easy to compute because it is a sum, and might only depend on sufficient statistics
% \item The EM algorithm implicitly deals with parameter constraints in the M-step
\item The EM algorithm is parameterization independent
\item The *Generalized* EM (GEM) just increases $Q$ instead of
  maximizing it.The book describes many generalizations including GEM.
\item You might find multiple modes if you start from multiple starting points (using mixture approximations afterwards)
\item if you can, debug by printing $\log p(\phi^i \mid y)$ at every iteration and make sure it increases monotonically
\end{enumerate}

\end{frame}

%----------------------------------------------------------------------------------------
\begin{frame}
  \frametitle{Marginal and conditional posterior density
    approximations}
  We can approximate $p(\phi \mid y)$ using normal, where the mode is
  found by the EM algorithm and we can numerically calculate the
  variance at the mode. 

  In other cases, it is not possible to construct an approximation to
  $p(\phi \mid y)$ using this method. We are going to discuss another possibility.
 
\end{frame}


%----------------------------------------------------------------------------------------
\begin{frame}
  \frametitle{Marginal and conditional posterior density
    approximations}
The approximation is constructed as follows:
\[
p_{\text{approx}}(\phi \mid y) = \frac{p(\gamma, \phi \mid
  y)}{p_{\text{approx}}(\gamma \mid \phi, y)}
\]

$p_{\text{approx}}(\gamma \mid \phi, y)$ is an analytic approximation
to $p(\gamma \mid \phi, y)$.

\begin{itemize}
\item If $p(\gamma \mid \phi, y)$ itself is analytic, any $\gamma$
  should not affect the distribution on the left hand side.
\item If the analytic approximation of the conditional distribution is
  not exact, we need to pick $\gamma$. The suggestion is use the
  center of the approximate distribution $p_{\text{approx}}(\gamma
  \mid \phi, y)$, denote it as $\hat{\gamma}(\phi)$.
\end{itemize}
If $p_{\text{approx}}(\gamma \mid \phi, y)$ is normal with mean
$\hat{\gamma}(\phi)$ and variance $V_\gamma(\phi)$,
$p_{\text{approx}}(\phi \mid y) \propto p(\hat{\gamma}(\phi), \phi
\mid y) |V_\gamma(\phi)|^{1/2}$.
\end{frame}


%----------------------------------------------------------------------------------------
\begin{frame}
  \frametitle{Marginal and conditional posterior density
    approximations}
Since $p(\phi \mid y) = \int p(\gamma, \phi \mid y) d\gamma$. We can also use the above trick and importance sampling to
approximate $p(\phi \mid y)$.
\[
p(\phi \mid y) = \int \frac{p(\gamma, \phi \mid y)}{p_{\text{approx}}(\gamma
  \mid \phi, y)} p_{\text{approx}}(\gamma
  \mid \phi, y) d\gamma 
\]

Sample $\gamma^1, \ldots, \gamma^S$ from $p_{\text{approx}}(\gamma
  \mid \phi, y)$, then $p(\phi \mid y) = (1/ S) \sum_{s=1}^S [p(\gamma^s, \phi \mid y)/p_{\text{approx}}(\gamma^s
  \mid \phi, y)]$
\end{frame}

\end{document} 

%%% Local Variables:
%%% mode: latex
%%% TeX-master: t
%%% End:
